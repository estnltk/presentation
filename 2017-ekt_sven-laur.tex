% !TEX encoding = UTF-8 Unicode
\documentclass[landscape,footrule]{foils}
\usepackage[lecture-serie]{foiltex-extra}
\usepackage{crysymb}
\usepackage[draft]{crygame}
\usepackage{graphics}
\usepackage[usenames,dvipsnames]{xcolor}
\usepackage[pdftex]{graphicx} 
\usepackage{mathtools}
\usepackage{url}
\usepackage[utf8]{inputenc}
\usepackage{soul}


\def\xcolorversion{2.00}
\def\xkeyvalversion{1.8}



\usepackage[version=0.96]{pgf}
\usepackage{tikz}
\usetikzlibrary{arrows,shapes,snakes,automata,backgrounds,petri}

\tikzset{
    V1/.style = {
        rectangle,
        rounded corners,
        minimum width = 7cm,
        minimum height = 1.5cm, 
        fill  = Apricot,
        draw = black,
        thick, 
    },
    V2/.style = {
        rectangle,
        rounded corners,
        minimum width = 7cm,
        minimum height = 1.5cm, 
        fill  = LimeGreen,
        draw = black,
        thick, 
    }
}


\newcommand{\lecture}{\ \vspace*{-5cm}\\EstNLTK muutuste tuules:\\ 1.4 $\rightsquigarrow$ 1.6}
\newcommand{\lserie}{}
\newcommand{\ldate}{20. oktoober 2015}
\newcommand{\lauthor}{\textbf{Sven Laur$^{\diamondsuit,\spadesuit}$}\vspace*{0ex}\\ \blue{\url{https://github.com/estnltk}} }
\newcommand{\linst}{\ \vspace*{1cm}\\ $^\diamondsuit$Tartu Ülikool\vspace*{1ex}\\ $^\spadesuit$Tarkvara Tehnoloogia Arenduskeskus OÜ}
\graphicspath{{./illustrations/}}
\MyLogo{Konverents "EESTI KEELETEHNOLOOGIA 2017", 18.–20. aprill 2017, Tallinn}


\newcommand{\colorcommand}[2]
{\hspace*{-0.15em}\colorbox{#1}{\hspace*{-0.25em}\ensuremath{#2}\rule{0pt}{1.6ex}\hspace*{-0.25em}}}
\newcommand{\lowcolorcommand}[2]
{\hspace*{-0.15em}\colorbox{#1}{\hspace*{-0.25em}\ensuremath{#2}\rule{0pt}{1.2ex}\hspace*{-0.25em}}}
\newcommand{\shared}[1]{[\![#1]\!]}

\newcommand{\leqm}{\ \leq_m}

\newcommand\redsout{\bgroup\markoverwith{\textcolor{red}{\rule[0.3ex]{2pt}{4.0pt}}}\ULon}


\newcommand{\bigvskip}{\vskip 2em}
\newcommand{\lastline}{\vspace*{-2ex}}
\newcommand{\spreadappart}{\vspace*{\fill}}

\DeclareMathOperator{\supp}{supp}
\DeclareMathOperator{\conf}{conf}
\DeclareMathOperator{\precision}{precision}
\DeclareMathOperator{\recall}{recall}


\begin{document}
\titlefoil



\foilhead[-1cm]{Mis on EstNLTK projekt?}
\LogoOn
\begin{triangles}
\item Vabavaraline Pythoni teek eestikeelsete tekstide töötlemiseks\vspace*{0.5ex} 
\begin{diamonds}
\item sõnestamine, lausestamine 
\item morfoloogiline ja süntaktiline analüüs
\item märgenduskihid fraaside eraldamiseks\vspace*{2ex}
\end{diamonds}
\item Eraldiseisvad rakendused standardsete probleemide lahendamiseks\vspace*{0.5ex} 
\begin{diamonds}
\item tekstide klassifitseerimine (\blue{textclassifier})
\item oluliste märksõnade tuvastamine (\blue{volcanoplot}) 
\item näitemärgenduste koostamine (\blue{ner-tagger}, \blue{gap-tagger}) \vspace*{2ex}
\end{diamonds} 
\item Keeleresursid ja testandmestikud
\vspace*{0.5ex} 
\begin{diamonds}
\item Lünktestid kontekstide ennustamiseks (\blue{word embeddings})
\item Nimeolemite ennustamise testandmestikud (\blue{named entity recognition}) 
\end{diamonds} 
\end{triangles}

\foilhead[-1cm]{EstNLTK 1.4 teek}

\begin{triangles}
\item Andmeformaat märgenduskihtide salvestamiseks\vspace*{0.5ex} 
\begin{diamonds}
\item tähe{\setlength{\fboxsep}{2pt}
\colorbox{Apricot}{põhi}\textcolor{Apricot}{\rule[3pt]{2em}{3pt}}\hspace*{-2em}sed \colorbox{Apricot}{märge}ndus\colorbox{LimeGreen}{elemendid\rule[-3pt]{0pt}{12pt}}} 
\item märgenduselemendile vastavad attribuudid
\item märgenduselementidest koosnevad märgenduskihid\vspace*{2ex}
\end{diamonds}
\item Liidesed olemasolevatele analüsaatoritele\vspace*{0.5ex} 
\begin{diamonds}
\item lausestaja
\item morfoanalüsaator (\blue{vabamorf})
\item pindsüntaktiline analüüs (\blue{visl}, \blue{maltparser})
\vspace*{2ex}
\end{diamonds}
\item Uued komponendid\vspace*{0.5ex} 
\begin{diamonds}
\item fraaside märgendajad (\blue{ajaväljendid}, \blue{verbiahelad}, \blue{$\ldots$})
\item keerulised fakti- ja fraasieraldusalgoritmid 
\vspace*{2ex}
\end{diamonds}
\end{triangles}

\foilhead[-1cm]{EstNLTK 1.4 märgenduskihid}

\begin{triangles}
\item \textbf{paragraphs}
\item \textbf{sentences}
\item \textbf{words} \vspace*{0.5ex} 
\begin{triangles}
\item \blue{analysis} \vspace*{0.5ex}
\begin{diamonds}
\item $\ldots$, form, lemma, $\ldots$\vspace*{1ex}
\end{diamonds} 
\item \blue{clause\_index} 
\item \blue{ner\_label} \vspace*{1ex}
\end{triangles}
\item \textbf{gt\_words} \vspace*{0.5ex} 
\begin{triangles}
\item \blue{analysis} \vspace*{0.5ex}
\begin{diamonds}
\item $\ldots$, form, lemma, $\ldots$\vspace*{2ex}
\end{diamonds}  
\end{triangles}
\item \textbf{verb\_chains}
\item \textbf{$\cdots$}
\end{triangles}

\foilhead[-1cm]{Puudused ja probleemid}
\begin{triangles}
\item Sõnadekiht (\textbf{words}) ja lausetekiht (\textbf{sentences}) võivad olla nihkes \\

\hspace*{3cm}
\begin{tabular}{ll}
\textbf{words:}&
{\setlength{\fboxsep}{3pt}
\colorbox{Apricot}{Ta} \colorbox{Apricot}{läks} \colorbox{Apricot}{kell} \colorbox{Apricot}{14\hspace*{5pt}.\hspace*{5pt}30} 
\colorbox{Apricot}{ära}}\\
\textbf{sentences:} &
{\setlength{\fboxsep}{3pt}
\colorbox{LimeGreen}{Ta\hspace*{6pt} läks\hspace*{6pt} kell\hspace*{6pt} 14.} \colorbox{LimeGreen}{30 \hspace*{3pt}ära} }\\
\end{tabular}\vspace*{2ex}


\item Osad kihid (\blue{analysis}, \blue{clauses}, $\ldots$) on esitatud atribuutidena \\ 

\hspace*{3.5cm}
\begin{minipage}{10cm} 
\begin{triangles}
\item \textbf{words} \vspace*{0.5ex} 
\begin{triangles}
\item \blue{analysis} \vspace*{0.5ex}
\begin{diamonds}
\item $\ldots$
\item form
\item  lemma
\item $\ldots$\vspace*{1ex}
\end{diamonds} 
\end{triangles}
\end{triangles}
\end{minipage}
\vspace*{1.5cm} 

kuid baasaoperatsioonid on defineeritud kihtidel  

\end{triangles}

\foilhead[-1cm]{Puudused ja probleemid}
\begin{triangles}
\item  Süsteemitus mitmesuste esitamisel. Kombinatoorne plahvatus \\
   
\hspace*{0.5cm}
\begin{minipage}{10cm} 
\begin{triangles}
\item \textbf{words} \vspace*{0.5ex} 
\begin{triangles}
\item \blue{analysis} \vspace*{1.5ex}\\
\begin{tabular}{cccc} 
\colorbox{Apricot}{
 \begin{minipage}{3.75cm}
 \begin{diamonds}
 \item $\ldots$
 \item form
 \item lemma
 \item $\ldots$\vspace*{1ex}
 \end{diamonds}
 \end{minipage}}
 &
 \colorbox{LimeGreen}{
 \begin{minipage}{3.75cm}
 \begin{diamonds}
 \item $\ldots$
 \item form
 \item lemma
 \item $\ldots$\vspace*{1ex}
 \end{diamonds} 
 \end{minipage}}
 & 
 $\cdots$
 &
 \colorbox{Tan}{
 \begin{minipage}{3.75cm}
 \begin{diamonds}
 \item $\ldots$
 \item form
 \item lemma
 \item $\ldots$\vspace*{1ex}
 \end{diamonds} 
 \end{minipage}}
 \\ \\
\colorbox{Apricot}{
 \begin{minipage}{3.75cm}
 \begin{diamonds}
 \item ner\_label
 \item $\ldots$\vspace*{1ex}
 \end{diamonds}
 \end{minipage}}
 &
 \colorbox{LimeGreen}{
 \begin{minipage}{3.75cm}
 \begin{diamonds}
 \item ner\_label
 \item $\ldots$\vspace*{1ex}
 \end{diamonds} 
 \end{minipage}}
 & 
 $\cdots$
 &
 \colorbox{Tan}{
 \begin{minipage}{3.75cm}
 \begin{diamonds}
 \item ner\_label
 \item $\ldots$\vspace*{1ex}
 \end{diamonds} 
 \end{minipage}}

\end{tabular}

\end{triangles}
\end{triangles}
\end{minipage}

\end{triangles}



\foilhead[-1cm]{EstNLTK 1.6 märgenduskihid}
\begin{triangles}
\item \textbf{words} \vspace*{0.5ex}
\begin{triangles}
\item \textbf{paragraphs}
\item \textbf{sentences}
\item \textbf{normalised\_words}
\item \textbf{morf\_analysis} \vspace*{0.5ex}
\begin{diamonds}
\item \blue{$\ldots$, form, lemma, $\ldots$}\vspace*{1ex}
\end{diamonds} 
\item \textbf{pronouns}\vspace*{0.5ex}
\begin{diamonds}
\item \blue{type}\vspace*{1ex}
\end{diamonds} 
\item \textbf{finite\_forms}\vspace*{0.5ex}
\begin{diamonds}
\item \blue{$\varnothing$}\vspace*{1ex}
\end{diamonds} 
\item \textbf{verb\_chains}

\item \textbf{$\cdots$} 
\end{triangles}
\item \textbf{$\cdots$}
\end{triangles}

\foilhead[-1cm]{Mitmesuste käsitlemine}
\enlargethispage{3cm}
\begin{triangles}
\item Kihis võib võib ühele märgenduselemendile vastata mitu tõlgendust\vspace*{2ex}

\begin{center}
{\setlength{\fboxsep}{3pt}
\colorbox{Apricot}{\rule[-3pt]{0pt}{16pt}tee}} $\to$
\begin{tabular}{|c|c|c|c|}
\hline
lemma & POS & form & $\ldots$\\
\hline
tee    & S & sg n & $\ldots$\\
tee    & S & sg g & $\ldots$\\
tegema & V &      & $\ldots$\\
\hline
\end{tabular}\vspace*{4ex}
\end{center} 
Iga elemendi tõlgendusele vastavad konkreetsed atribuutide väärtused\vspace*{2ex}
\item Erinevate elementide tõlgendused kihis on kombineeruvad sõltumatult \vspace*{2ex}

\begin{center}
$\left(
\begin{minipage}{6cm}
{\setlength{\fboxsep}{3pt}
\colorbox{Apricot}{tee\rule[-3pt]{0pt}{16pt}}} $\to$
\begin{tabular}{|c|}
\hline
lemma \\
\hline
tee  \\
tegema \\
\hline
\end{tabular}
\end{minipage}\right)
\quad\times\quad
\left(
\begin{minipage}{6cm}
{\setlength{\fboxsep}{3pt}
\colorbox{Apricot}{aja}} $\to$
\begin{tabular}{|c|}
\hline
lemma \\
\hline
aeg  \\
ajama \\
\hline
\end{tabular}
\end{minipage}
\right)
$

\end{center}

\foilhead[-1cm]{Mitmesuste käsitlemine}
\enlargethispage{1cm}

\item Erinevate kihtide tõlgendused on kombineeruvad sõltumatult  \vspace*{2ex}
\begin{align*}
\left(
\begin{minipage}{9cm}
\textbf{morf:}\ 
{\setlength{\fboxsep}{3pt}
\colorbox{Apricot}{tee\rule[-3pt]{0pt}{16pt}}} 
$\to$
\begin{tabular}{|c|c|c|c|}
\hline
 \ \ lemma\ \;  \\
\hline
tee    \\
tee    \\
tegema \\
\hline
\end{tabular}
\end{minipage}\right)\\
\times \hspace*{4.5cm}\\
\left(
\begin{minipage}{9cm}
\textbf{ner:}\;\; 
{\setlength{\fboxsep}{3pt}
\colorbox{Apricot}{tee\rule[-3pt]{0pt}{16pt}}} 
 $\to$
\begin{tabular}{|c|}
\hline
\hspace{0.7cm}
type\hspace{0.7cm} \\
\hline
LOC \\
ORG \\
\hline
\end{tabular}
\end{minipage}
\right)
\end{align*} 
\end{triangles}



\foilhead[-1cm]{Robustne tekstianalüüs töövoog}
\begin{center}
\hspace*{-18cm}
\begin{tikzpicture}[transform canvas={scale=0.65}]
  \draw node[V2] (S1) at ( 9, 0) {Sõnestamine};
  \draw node[V1] (S2) at (18,-2) {Järelsõnestamine};
  \draw node[V2] (L1) at ( 9,-4) {Lausestamine};
  \draw node[V1] (L2) at (18,-6) {Järellausestamine};
  \draw node[V1] (M0) at ( 0,-8) {Normaliseerimine};
  \draw node[V2] (M1) at ( 9,-10) {Morf. analüüs};
  \draw node[V1] (M2) at (18,-12) {Morf. parandused};
  \draw node[V2] (PS1) at ( 9,-14) {Morf. teisendused};
  \draw node[V1] (PS2) at (18,-16) {Vigade diagnostika};
  \draw node[V2] (VS1) at (27,-16) {VISL};
  
  \draw[->, line width=3pt] (S1) -| (S2);
  \draw[->, line width=3pt] (S2) -| (L1);
  \draw[->, line width=3pt] (L1) -| (L2);
  \draw[->, line width=3pt] (L2) -| (M0);
  \draw[->, line width=3pt] (M0) -| (M1);
  \draw[->, line width=3pt] (M1) -| (M2);
  \draw[->, line width=3pt] (M2) -| (PS1);
  \draw[->, line width=3pt] (PS1) -| (PS2);
  \draw[->, line width=3pt] (PS1) -| (VS1);

\end{tikzpicture}
\end{center}

\foilhead[-1cm]{Milleks on sellist analüüsi vaja?}

\textbf{Probleem:} Koondkorpuses on üle 10 000 erineva asesõna
\begin{triangles}
\item tege-ma $\to$ lemma: tege-mina
\item 1203-me $\to$ lemma: 1203-meie
\item kes-teab-mis $\to$ lemma: kes-teab-mis
\end{triangles}
\vspace*{3cm}
Olemasolev töövoog ei sisalda standardseid teksti puhastussamme
\begin{triangles}
\item See muudab raskemaks tüüpiliste tekstide analüüsi
\item Iga üks meist leiutab samu tekstide puhastamise võtteid
\item Ei jää aega korpuse spetsiifiliste probleemide tuvastamiseks 
\end{triangles}




\foilhead[-1cm]{Fraaside eraldamine}

\begin{center}
Pulss neljapäeval 14.13:56 lööki\vspace*{2ex}
\begin{tabular}{|cccccccc|}
\hline
Pulss & neljapäeval & 14 & . &13 &: &56 &lööki\rule[-1ex]{0pt}{1ex}\\
\hline
\hline
      & \colorbox{Apricot}{\hspace{0.5em}\smash{DATE}\rule{0pt}{12pt}\hspace{0.5em}}       
      & \multicolumn{5}{l}{\colorbox{LimeGreen}{\hspace{5.0em}\smash{TIME}\rule{0pt}{12pt}\hspace{5.0em}}} \rule[-2ex]{0pt}{5ex} &\\
      &
      & \multicolumn{3}{c}{\colorbox{LimeGreen}{\hspace{1.7em}\smash{TIME}\rule{0pt}{12pt}\hspace{2.0em}}}
      &
      & \colorbox{Tan}{\hspace*{0.25em}\smash{NUM}\rule{0pt}{12pt}\hspace*{0.25em}}
      & \colorbox{Dandelion}{\hspace*{0.25em}\smash{UNIT}\rule{0pt}{12pt}\hspace*{0.25em}}
      \rule[-2ex]{0pt}{1ex}\\
\hline      
\end{tabular}\vspace*{1cm}
\end{center}
\begin{triangles}
\item Huvitavad fraasid on kirjeldatavad \emph{lõpliku} grammatikaga
\item Fraaside parsimisel olev sõnestus pole ühene ega ülekatteta
\item Sõnestuse ühestamiseks on mõistlik kasutada grammatikat
\item Reeglite prioriteete on kasulik annoteerida tõenäosustega 
\end{triangles}


\foilhead[-1cm]{Kokkuvõte}

\textbf{Hea}
\begin{triangles}
\item EstNLTK liides ühtlustub 
\item EstNLTK kasutamine muutub ettearvatavamaks
\item EstNLTK liidestub erinevate veebikeskkondadega
\item EstNLTK katab operatsioonid sõnestusest kuni süntaksianalüüsini
\end{triangles}
\vspace*{2cm}
\textbf{Halb}
\begin{triangles}
\item EstNLTK liides liides muutub
\item EstNLTK nõuab Pythoni versiooni 3.5
\item Kõik EstNLTK 1.4 olevad komponendid ei jõua EstNLTK 1.6
\end{triangles}



\foilhead[-1cm]{Täname}

\textbf{Arendajad}
\begin{center}
\begin{tabular}{cc}
\begin{minipage}[t]{7cm}
\begin{bullets}
\item Siim Orasmaa
\item Timo Petmanson
\item Uku Raudvere
\end{bullets}
\end{minipage}
&
\begin{minipage}[t]{10cm}
\begin{bullets}
\item Dage Särg
\item Paul Tammo
\item Aleksandr Tkatšenko
\end{bullets}  
\end{minipage}
\end{tabular}
\end{center}
\vspace*{2cm}
\textbf{Nõustajad}
\begin{center}
\begin{minipage}[t]{10cm}
\begin{bullets}
\item Heiki-Jaan Kaalep
\item Kadri Muischnek
\item Tarmo Vaino
\end{bullets}  
\end{minipage}
\hspace*{7.5cm}
\end{center}




\end{document}


