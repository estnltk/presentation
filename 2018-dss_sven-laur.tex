% !TEX encoding = UTF-8 Unicode
\documentclass[landscape,footrule]{foils}
\usepackage[lecture-serie]{foiltex-extra}
\usepackage{crysymb}
\usepackage[draft]{crygame}
\usepackage{graphics}
\usepackage[usenames,dvipsnames]{xcolor}
\usepackage[pdftex]{graphicx} 
\usepackage{mathtools}
\usepackage{url}
\usepackage[utf8]{inputenc}
\usepackage{soul}



\def\xcolorversion{2.00}
\def\xkeyvalversion{1.8}



\usepackage[version=0.96]{pgf}
\usepackage{tikz}
\usetikzlibrary{arrows,shapes,snakes,automata,backgrounds,petri}

\tikzset{
    V1/.style = {
        rectangle,
        rounded corners,
        minimum width = 7cm,
        minimum height = 1.5cm, 
        fill  = Apricot,
        draw = black,
        thick, 
    },
    V2/.style = {
        rectangle,
        rounded corners,
        minimum width = 7cm,
        minimum height = 1.5cm, 
        fill  = LimeGreen,
        draw = black,
        thick, 
    }
}


\newcommand{\lecture}{EstNLTK libraries for NLP}
\newcommand{\lserie}{}
\newcommand{\ldate}{20. oktoober 2015}
\newcommand{\lauthor}{\textbf{Sven Laur$^{\spadesuit,\diamondsuit}$}\vspace*{1ex}\\ \blue{\url{https://github.com/estnltk}} }
\newcommand{\linst}{\ \vspace*{1cm}\\ $^\spadesuit$STACC\\ 
$^\diamondsuit$University of Tartu\vspace*{1ex} }
\graphicspath{{./illustrations/}}
\MyLogo{Data Science seminar, EstNLTK libraries for NLP, 26 September 2018, Tartu}


\newcommand{\colorcommand}[2]
{\hspace*{-0.15em}\colorbox{#1}{\hspace*{-0.25em}\ensuremath{#2}\rule{0pt}{1.6ex}\hspace*{-0.25em}}}
\newcommand{\lowcolorcommand}[2]
{\hspace*{-0.15em}\colorbox{#1}{\hspace*{-0.25em}\ensuremath{#2}\rule{0pt}{1.2ex}\hspace*{-0.25em}}}
\newcommand{\shared}[1]{[\![#1]\!]}

\newcommand{\leqm}{\ \leq_m}

\newcommand\redsout{\bgroup\markoverwith{\textcolor{red}{\rule[0.3ex]{2pt}{4.0pt}}}\ULon}


\newcommand{\bigvskip}{\vskip 2em}
\newcommand{\lastline}{\vspace*{-2ex}}
\newcommand{\spreadappart}{\vspace*{\fill}}

\DeclareMathOperator{\supp}{supp}
\DeclareMathOperator{\conf}{conf}
\DeclareMathOperator{\precision}{precision}
\DeclareMathOperator{\recall}{recall}


\begin{document}
\titlefoil

\foilhead[-1cm]{Yet another library for NLP?}
\LogoOn

There are myriad of libraries for processing text
\begin{triangles}
\item GATE, Stanford CoreNLP (Java, general purpose) 
\item NLTK, SPACY (Python, general purpose)
\item Scikit-learn, Gensim (Python, specific tasks)\vspace*{1cm}
\end{triangles}

There are many tools for analysing Estonian language
\begin{triangles}
\item Vabamorf, EstHST, EstNeuroMorph (morphology)
\item EstCG, EstMalt (syntax) 
\item Estonian WordNet
\item Named Entity Recognition
\item $\ldots$
\end{triangles} 

\foilhead[-1cm]{Goals of EstNLTK project}

EstNLTK is a Python library distributed under GPLv2 license: 
\begin{triangles}
\item Easy to install, learn and use  
\item Unified framework for text annotations 
\item Programmatic access to existing text analysis tools
\item Predefined but reconfigurable workflows for common tasks
\item Unified framework for visualisation and storing annotations\vspace*{1cm}
\end{triangles}

\foilhead[-1cm]{How can I use EstNLTK?}

Installation
\begin{triangles}
\item Anaconda binary packages (easy)
\item Standard pip installer (I like obscure C++ compiling errors)
\item The latest GIT commit (Random hacks to avoid C++ compiler)\vspace*{1cm}
\end{triangles}


Licensing
\begin{triangles}
\item I believe in free software (GPLv2)
\item I am building a service or software for internal usage (GPLv2)
\item I need more liberal licence to sell a commercial product:\vspace*{0.5ex}
\begin{diamonds}
\item Contact University of Tartu for different license
\item Write a wrapper to make EstNLTK as a separate process
\end{diamonds}
\end{triangles}

\foilhead[-1cm]{Basic concepts}

\begin{triangles}
\item Annotations for text spans (\textsc{Taggers})\\
\centerline{\includegraphics[scale=0.9]{annotations}}
\item Annotations for relations between spans (\textsc{Syntax} \& \textsc{Complex facts})\\
\centerline{\includegraphics[scale=0.9]{syntax}}
\end{triangles}

\foilhead[-1cm]{Basic concepts}

\begin{triangles}
\item Annotations for relations between spans (\textsc{Syntax} \& \textsc{Complex facts})\\
\centerline{\includegraphics[scale=0.9]{syntax}}
\item Span trees on top of text spans (\textsc{Phrases} \& \textsc{Fact extraction})\\
\centerline{\includegraphics[scale=0.9]{trees}}
\end{triangles}


\foilhead[-1cm]{What can you tag with EstNLTK?}
\begin{triangles}
\item Basic building blocks of text:\vspace*{0.5ex} 
\begin{diamonds}
\item words, sentences, paragraphs.\vspace*{2ex} 
\end{diamonds}
\item Morphology\vspace*{0.5ex} 
\begin{diamonds}
\item lemma, part of speech, case, tense, number,$\ldots$
\item in \textcolor{blue!100}{\textbf{Estmorf}}, \textcolor{blue!70}{Giellatekno}, \textcolor{blue!40}{Visl CG} formats,  \vspace*{2ex} 
\end{diamonds}
\item Important phrases (\textcolor{orange!80!black!60}{\textbf{EstNLTK 1.4}})
\vspace*{0.5ex} 
\begin{diamonds}
\item named entities
\item noun phrases, adjective phrases, verb chains 
\item clauses, time phrases\vspace*{2ex}  
\end{diamonds}

\item \textcolor{gray!70}{Syntax (EstNLTK 1.4)}
\vspace*{0.5ex} 
\begin{diamonds}
\item \textcolor{gray!70}{EstCG and EstMalt models}
\end{diamonds}
\end{triangles}

\foilhead[-1cm]{What does EstNLTK 1.6 offer?}

\begin{triangles}
\item \textcolor{green!50!black!100}{Span hierarchies}
\item \textcolor{green!50!black!100}{Ambiguous annotations}

\item \textcolor{orange!90}{New analysis algorithms}
\item \textcolor{green!50!black!100}{Predefined analysis workflows}
\item \textcolor{green!50!black!100}{Two-phase fact extraction algorithms}
\item \textcolor{green!50!black!100}{Postgre collections for storage and search}
\item \textcolor{orange!90}{Better visualisation and integration with jupyter}
\item \textcolor{orange!90}{Standard taggers for important phrases}
\item \textcolor{red!90!black!100}{Native support for syntax analysis}

\end{triangles}

\foilhead[-1cm]{Span hierarchies in EstNLTK 1.6}

\centerline{\includegraphics[scale=0.9]{span-hierachy}}

It is natural to define new spans in terms of other spans:
\begin{diamonds}
\item new phrases 
\item independent annotations
\end{diamonds}

\foilhead[-1cm]{Ambiguous annotations}

\centerline{\includegraphics[scale=0.9]{ambiguous}}

Sometimes it is impossible to assign unique interpretation to a span
\begin{diamonds}
\item A span can have several annotations
\item Annotations of different spans are independent
\end{diamonds}

\foilhead[-1cm]{Robust NLP pipleline}

\textbf{Why?}
\begin{triangles}
\item Good analysis requires many tedious cleaning steps 
\item You can incorporate data specific tweaks into the pipeline\vspace*{1cm}
\end{triangles}


\textbf{What does the robust NLP pipeline do?}
\begin{triangles}
\item Identifies compound tokens:\vspace*{0.5ex}
\begin{diamonds}
\item numbers, dates, units, urls, emails, xml-tags 
\item abbreviations, emoticons, symbol tokens, compound names,$\ldots$\vspace{1ex}
\end{diamonds}
\item Identifies normal forms for words:\vspace*{0.5ex}
\begin{diamonds}
\item date normalisation
\item \textcolor{red!90!black!100}{corrects spelling mistakes}\vspace{1ex}
\end{diamonds}
\item $\ldots$
\end{triangles}

\foilhead[-1.5cm]{What does the robust NLP pipeline do?}
\enlargethispage{1cm}
\begin{triangles}
\item Identifies compound tokens:\vspace*{0.5ex}
\begin{diamonds}
\item numbers, dates, units, urls, emails, xml-tags 
\item abbreviations, emoticons, symbol tokens, compound names,$\ldots$\vspace{1ex}
\end{diamonds}
\item Identifies normal forms for words:\vspace*{0.5ex}
\begin{diamonds}
\item date normalisation
\item \textcolor{red!90!black!100}{corrects spelling mistakes}\vspace{1ex}
\end{diamonds}
\item Identifies sentence and paragraph borders\vspace*{0.5ex}
\begin{diamonds}
\item standard sentence detection
\item post-corrections (numbers, abbreviations, emoticons)\vspace{1ex}
\end{diamonds}
\item Performs morphological analysis \vspace*{0.5ex}
\begin{diamonds}
\item vabamorf
\item post-corrections for specific words (number-text combos)\vspace{1ex}
\end{diamonds}
\item \textcolor{orange}{Performs syntax analysis}
\end{triangles}


\foilhead[-1cm]{Two-phase fact extraction}

\centerline{\includegraphics[scale=0.9]{psa-example}}

Fact extraction can be done with finite attribute grammars
\begin{triangles}
\item Tokenisation is often ambiguous
\item Grammar rules filter out spurious token variants
\item Meaning can be given iteratively from bottom up\vspace*{1cm}
\end{triangles}

\foilhead[-1cm]{Tweaks to the previous ideas}

\begin{triangles}
\item It is convenient to allow infinite sequences 

\centerline{\includegraphics[scale=0.9]{sequence-example}}
\item It is convenient to add rule priorities

\centerline{\includegraphics[scale=0.9]{priority-example}}

\item It is convenient to cancel productions with conflicting attributes

\centerline{\includegraphics[scale=0.9]{attribute-conflict}}


\end{triangles}


\foilhead[-1cm]{Postgre storage}

\textbf{Why?}
\begin{triangles}
\item Provides a simple searchable JSON serialisation to EstNLTK objects
\item Allows to speeds up fact extraction after tokenisation is done\vspace*{1.0cm} 
\end{triangles}

\textbf{What does Postgre collection do?}
\begin{triangles}
\item Allows to define new layers 
\item Allows to index layers with fingerprints
\item Allows to compare different layers during grammar development\vspace*{1.0cm} 
\end{triangles}


\textbf{What does Postgre collection contain?}
\begin{triangles}
\item Text objects, outer layers, layer fragments 
\item Meta attributes for text objects and layers
\end{triangles}


\foilhead[-1cm]{Contributors to EstNLTK project}

\textbf{Developers}
\begin{center}
\begin{tabular}{cc}
\begin{minipage}[t]{7cm}
\begin{bullets}
\item Siim Orasmaa
\item Timo Petmanson
\item Uku Raudvere
\end{bullets}
\end{minipage}
&
\begin{minipage}[t]{10cm}
\begin{bullets}
\item Dage Särg
\item Paul Tammo
\item Aleksandr Tkatšenko
\end{bullets}  
\end{minipage}
\end{tabular}
\end{center}
\vspace*{2cm}
\textbf{Consulting}
\begin{center}
\begin{minipage}[t]{10cm}
\begin{bullets}
\item Heiki-Jaan Kaalep
\item Kadri Muischnek
\item Kairit Sirts
\item Tarmo Vaino
\end{bullets}  
\end{minipage}
\hspace*{7.5cm}
\end{center}


\end{document}






\end{document}


